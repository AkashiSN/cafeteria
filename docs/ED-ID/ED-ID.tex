\documentclass[a4paper]{ltjsarticle}
\usepackage{graphicx}
\usepackage{enumitem}
\usepackage{siunitx}
\usepackage{multirow}
\usepackage{cprotect}

\setlist[enumerate,1]{label= \textcircled{\scriptsize \arabic*}}
\setlist[enumerate,2]{label= \arabic*.}

%% Hyphenation setting
\hyphenpenalty=1000\relax
\exhyphenpenalty=1000\relax
\sloppy

\makeatletter
\def\Hline{
    \noalign{\ifnum0=`}\fi\hrule \@height 2.\arrayrulewidth \futurelet
    \reserved@a\@xhline}
\makeatother

\begin{document}
% ------------------------------------------------------
% front cover
% ------------------------------------------------------

\large
\vspace{-5.0cm}
\hspace{-1.0cm}
情報工学実験II

\hspace{-1.0cm}
2019年5月8日

\Huge
\vspace{1.0cm}
\begin{center}
    食堂システム 外部・内部設計書
\end{center}

\vspace{0.5cm}
\begin{center}
    \LARGE
    明石工業高等専門学校
\end{center}

\LARGE
\begin{center}
    \begin{tabular}{rl}
        E1507 & 泉 和哉 \\
        E1514 & 岡本 一真 \\
        E1533 & 西 総一朗
    \end{tabular}
\end{center}

\normalsize

\tableofcontents
\thispagestyle{empty}
\clearpage
\setcounter{page}{1}

% ------------------------------------------------------
% text start
% ------------------------------------------------------

\section{外部設計}
\subsection{システム}
    \subsubsection{サーバー構成}
        表\ref{tab:server_hardware},\,\ref{tab:server_software}にサーバーの構成を示す.
        \begin{table}[h]
            \begin{minipage}[t]{.49\textwidth}
                \center
                \caption{ハードウェア構成}
                \label{tab:server_hardware}
                \begin{tabular}{|c||c|}
                    \hline
                    \multicolumn{2}{|c|}{サーバーのハードウェア} \\ \hline \hline
                    Model & HP-EliteDesk 800 G1 SFF \\ \hline
                    CPU & Intel\textregistered\, Core\texttrademark\, i7-4790 $\SI{3.6}{[\giga\hertz]}$ \\ \hline
                    Memory & 16 [GB]\\ \hline
                    SSD & 256 [GB] \\ \hline
                \end{tabular}
            \end{minipage}
            \begin{minipage}[t]{.49\textwidth}
                \center
                \caption{ソフトウェア構成}
                \label{tab:server_software}
                \begin{tabular}{|c||c|}
                    \hline
                    \multicolumn{2}{|c|}{サーバーのソフトウェア} \\ \hline \hline
                    OS & Ubuntu 16.04.6 LTS \\ \hline
                    Apache & 2.4.18 (Ubuntu) \\ \hline
                    PostgreSQL & 9.5.16 \\ \hline
                    PHP & 7.0.33-0ubuntu0.16.04.4 \\ \hline
                    Perl & v5.22.1 \\ \hline
                \end{tabular}
            \end{minipage}
        \end{table}
    \subsubsection{クライアント構成}
        表\ref{tab:client_hardware},\,\ref{tab:client_software}にクライアントの構成を示す.
    \subsubsection{ネットワーク}
        表\ref{tab:network}にネットワークの構成を示す.
        \begin{table}[h]
            \begin{minipage}[t]{.6\textwidth}
                \center
                \caption{ハードウェア構成}
                \label{tab:client_hardware}
                \begin{tabular}{|c||c|}
                    \hline
                    \multicolumn{2}{|c|}{クライアントのハードウェア} \\ \hline \hline
                    Model & ASUS ZenPad Z581KL \\ \hline
                    Display & 7.9 inch (QXGA) (2048\times 1536) \\ \hline
                    CPU & Qualcomm\textregistered\, Snapdragon\texttrademark\, 650 \\ \hline
                    GPU & Adreno\texttrademark\, 510 \\ \hline
                    Memory & 4 [GB]\\ \hline
                    Storage & 32 [GB] \\ \hline
                    \multirow{2}{*}{Wireless} &  IEEE 802.11ac/n/a/g/b \\
                    & Bluetooth 4.1 \\ \hline
                    \multirow{3}{*}{Sensor} & GPS(GLONASSサポート),加速度センサ, \\
                    & 光センサ,電子コンパス,磁気センサ, \\
                    & 近接センサ,ジャイロセンサ \\ \hline
                \end{tabular}
            \end{minipage}
            \begin{minipage}[t]{.39\textwidth}
                \center
                \caption{ソフトウェア構成}
                \label{tab:client_software}
                \begin{tabular}{|c||c|}
                    \hline
                    \multicolumn{2}{|c|}{クライアントのソフトウェア} \\ \hline \hline
                    OS & Android\texttrademark\, 7.0 \\ \hline
                    Chrome & 56.0.2924.87 \\ \hline
                \end{tabular}
                \hspace{1\textwidth}
                \caption{ネットワーク構成}
                \label{tab:network}
                \begin{tabular}{|c||c|}
                    \hline
                    \multicolumn{2}{|c|}{ネットワーク} \\ \hline \hline
                    ホスト名 & radish6.knet \\ \hline
                    IPアドレス & 172.16.16.7 \\ \hline
                \end{tabular}
            \end{minipage}
        \end{table}

\subsection{セキュリティ対策}
    \subsubsection{不正アクセス}

\subsection{ユーザーインターフェース}
    \subsubsection{画面遷移図}
    \subsubsection{画面}

\section{内部設計}
    \subsection{サーバサイド技術}
        サーバサイド技術はPHP\,7.0を使用し,フレームワークとしてLaravel\,5.5を使用する.
        そのため,\, PHP拡張モジュールとして以下が必要となる.
        \begin{itemize}
            \item OpenSSL PHP Extension
            \item PDO PHP Extension
            \item Mbstring PHP Extension
            \item Tokenizer PHP Extension
            \item XML PHP Extension
            \item JSON PHP Extension
            \item PostgreSQL PHP Extension
        \end{itemize}
    \subsection{データベース}
        \begin{table}[h]
            \begin{minipage}[t]{.49\textwidth}
                \center
                \caption{日替わりメニュー}
                \label{daily-menu}
                \begin{tabular}{|c|c|}
                    \hline
                    \multicolumn{2}{|c|}{\texttt{daily\_menu}} \\ \hline \hline
                    \verb|date| & \verb|DATE| \\ \Hline
                    \verb|menu_id_A| & \verb|INTEGER| \\ \hline
                    \verb|menu_id_B| & \verb|INTEGER| \\ \hline
                \end{tabular}
                \center
                \caption{売り切れ}
                \label{soldout}
                \begin{tabular}{|c|c|}
                    \hline
                    \multicolumn{2}{|c|}{\texttt{sold\_out}} \\ \hline \hline
                    \verb|menu_id| & \verb|INTEGER| \\ \Hline
                    \verb|sold_out| & \verb|BOOLEAN| \\ \hline
                    \verb|created_at| & \verb|TIMESTAMP| \\ \hline
                    \verb|updated_at| & \verb|TIMESTAMP| \\ \hline
                \end{tabular}
            \end{minipage}
            \begin{minipage}[t]{.49\textwidth}
                \center
                \caption{メニュー}
                \label{menu}
                \begin{tabular}{|c|c|}
                    \hline
                    \multicolumn{2}{|c|}{\texttt{menus}} \\ \hline \hline
                    \verb|menu_id| & \verb|BIGINT| \\ \Hline
                    \verb|item_name| & \verb|VARCHAR(255)| \\ \hline
                    \verb|category| & \verb|VARCHAR(255)| \\ \hline
                    \verb|price| & \verb|INTEGER| \\ \hline
                    \verb|energy| & \verb|DOUBLE PRECISION| \\ \hline
                    \verb|protein| & \verb|DOUBLE PRECISION| \\ \hline
                    \verb|lipid| & \verb|DOUBLE PRECISION| \\ \hline
                    \verb|salt| & \verb|DOUBLE PRECISION| \\ \hline
                \end{tabular}
            \end{minipage}
        \end{table}
        \begin{table}[ht]
            \begin{minipage}[t]{.49\textwidth}
                \center
                \caption{レビュー}
                \label{reviews}
                \begin{tabular}{|c|c|}
                    \hline
                    \multicolumn{2}{|c|}{\texttt{reviews}} \\ \hline \hline
                    \verb|review_id| & \verb|BIGINT| \\ \Hline
                    \verb|menu_id| & \verb|INTEGER| \\ \hline
                    \verb|evaluation| & \verb|INTEGER| \\ \hline
                    \verb|comment| & \verb|VARCHAR(255)| \\ \hline
                    \verb|image_path| & \verb|VARCHAR(255)| \\ \hline
                    \verb|created_at| & \verb|TIMESTAMP| \\ \hline
                    \verb|updated_at| & \verb|TIMESTAMP| \\ \hline
                \end{tabular}
            \end{minipage}
            \begin{minipage}[t]{.49\textwidth}
                \center
                \caption{ユーザー}
                \label{users}
                \begin{tabular}{|c|c|}
                    \hline
                    \multicolumn{2}{|c|}{\texttt{users}} \\ \hline \hline
                    \verb|user_id| & \verb|BIGINT| \\ \Hline
                    \verb|name| & \verb|VARCHAR(255)| \\ \hline
                    \verb|email| & \verb|VARCHAR(255)| \\ \hline
                    \verb|created_at| & \verb|TIMESTAMP| \\ \hline
                    \verb|updated_at| & \verb|TIMESTAMP| \\ \hline
                \end{tabular}
            \end{minipage}
        \end{table}

\begin{thebibliography}{9}
    \bibitem{doc} 本実験資料
\end{thebibliography}

\end{document}
